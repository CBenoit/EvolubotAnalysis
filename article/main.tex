\documentclass[sigconf, authordraft, anonymous]{acmart}

\usepackage[utf8]{inputenc}

%% For algorithms
\usepackage{algorithm}
\usepackage{algpseudocode}

%% For bibtex
\usepackage{natbib}

%% for good tables
\usepackage{array}
\usepackage{booktabs}

% \setlength{\parskip}{0.3em}

% Misc to use bépo keyboard.
\DeclareUnicodeCharacter{00A0}{~}
\DeclareUnicodeCharacter{00D7}{\times}

% Helper commands
\newcommand{\tl}{\textless}
\newcommand{\tg}{\textgreater}

\newcommand{\bb}[1]{\mathbb{#1}} % the blackboard bold capital is used for sets.



\setcopyright{rightsretained}

% DOI
\acmDOI{10.1145/nnnnnnn.nnnnnnn}

% ISBN
\acmISBN{978-x-xxxx-xxxx-x/YY/MM}

% Conference
\acmConference[GECCO '18]{the Genetic and Evolutionary Computation
Conference 2018}{July 15--19, 2018}{Kyoto, Japan}
\acmYear{2018}
\copyrightyear{2018}

\acmPrice{15.00}

\acmSubmissionID{123-A12-B3}

\begin{document}

\title{Humane approach to Starcraft micromanagement}

\author{Benoît Cortier}
\affiliation{\institution{University of Technology of Belfort-Montbéliard}
             \streetaddress{90010 cedex}
             \city{Belfort}
             \country{France}
             }
\affiliation{\institution{University of Tsukuba, Graduate School of Systems and Information Engineering}
             \streetaddress{Tennodai 1-1-1}
             \city{Tsukuba City}
             \country{Japan}
             }
\email{benoit.cortier@fried-world.eu}

\author{Claus Aranha}
\affiliation{\institution{University of Tsukuba, Faculty of Engineering, Information and Systems}
             \streetaddress{Tennodai 1-1-1}
             \city{Tsukuba City}
             \country{Japan}
             }
\email{caranha@cs.tsukuba.ac.jp}

% 174 words (Maximum 200 words)
\begin{abstract}
We study the use of NEAT and variants to develop strategies for the
micromanagement of squads of units in the Starcraft: Brood Wars game.
While current research in this field focuses on developing strategies
that maximize win rate, we are interested in also minimizing the amount
of losses in terms of destroyed units. This is important when one considers
that Starcraft battles are in the context of a longer strategic play session,
and in this sense conserving resources is important. To achieve this goal,
we defined a fitness function that takes into account the number of
surviving units, and experimented with four NEAT variants: vanilla NEAT,
novelty-search NEAT, unified (or homogeneous) NEAT and cascade NEAT. We
explored these four variants on with four different units match ups. The
four variants performed similarly on win ratio and survival metrics.
However, we performed a qualitative analysis as well and found that differents
behaviours emerged or that they emerged in different ways. From all that we draw
ideas for future work that could achieve our original goal.
\end{abstract}

\begin{CCSXML}
<ccs2012>
<concept>
<concept_id>10010405.10010476.10011187.10011190</concept_id>
<concept_desc>Applied computing~Computer games</concept_desc>
<concept_significance>500</concept_significance>
</concept>
</ccs2012>
\end{CCSXML}

\ccsdesc[500]{Applied computing~Computer games}

\keywords{Starcraft, micromanagement, NEAT, neuroevolution, artificial
          neural networks, evolutionary algorithms, RTS}

\maketitle

\begin{figure}
  \includegraphics[width=.45\textwidth]{figures/vultures_vs_zealots_combat}
  \caption{Our AI controller leading vultures (right) against zealots
    (left). Our goal is to win small scale combats like this while
    reducing the number of lost units}
    \label{fig:combat-example}
\end{figure}

\section{Introduction}\label{section:introduction}

\emph{Starcraft: Brood Wars} is a Real Time Strategy (RTS) computer
game that has in recent years captured the attention of researchers on
game-playing artificial intelligence. Starcraft is a complex game,
including the fact that incomplete information is available for each
player, and that its actions are stochastic. Therefore it is a hard
and interesting challenge for Artificial Intelligence. Also it can be
approached from several levels, such as Strategic Planning, Execution
of tactical maneuvers, Estimation of hidden information, etc.

In this work, we focus on the problem of ``micro'' in
Starcraft. Micro, short for micromanagement, is the problem of
directly controlling a small number of units directly in an engagement
with units from an opponent player. The tasks required in micro
include moving damaged units out of the fire range from the enemy,
spreading the units to maximize the fire arc, and using an unit's
special abilities at the right time. Also, the decisions for each of
these tasks will differ according to the units controlled by the
player and the opponent.

Learning approaches for micro in Starcraft and other similar games has
been studied by many groups recently. For example, \citet{Shantia11ConnectionistSC}
created a micro controller focused
around a neural network with weights adjusted by Sarsa;
\citet{Wender12ReinforcementMicroSC} explored Q-learning and Sarsa;
\citet{Zhen13NeuroEvoSC} compare NEAT and rtNEAT; and \citet{Liu14EffectiveMicro}
influence maps and potential fields.  Although each of these works uses a different
algorithm to generate the micro controller, one thing that they all
have in common is that they mostly measure success as the win rate,
i.e. the total number of combats won (although \citet{Liu14EffectiveMicro} does include
some measure of unit loss and survivors in their evaluation).

In this work we want to explore how to evolve a micro controller that
is able to not only achieve a high win rate, but also do so with as
few losses as possible. In an actual game, it is important for a player
to preserve their own forces, so that they can reallocate the remaining
forces to future battle, and compound their advantage through the match.

To achieve this, we propose an evaluation function that takes into
account the damage taken and the number of surviving units at the end
of a match, and compare four different NEAT implementations using this
utility function, as well as Novelty Search. We analyze the results of
four Match ups (Marines vs Marines, Marines vs Zerg, Vultures vs
Vultures and Vultures vs Zealots). Quantitatively, none of the
implemented variants showed a strong difference from current
results. But a qualitative analysis indicate directions in the way
behaviours emerged.


\section{Background}\label{section:background}

\subsection{Starcraft}

Starcraft is a \emph{Real Time Strategy} (RTS) game published by
Blizzard Entertainment. It follows the ``gather-build-conquer''
pattern, in which each player (human or AI) must gather resources
available in the play field to build combat units and
structures. These units are used to attack the other players,
destroying their units and structures and denying their access to
resources. The game features a wide variety of combat units, with
different speeds, attack range, strengths and weakenesses. The balance
between combat and resource gathering leads to the concepts of
\emph{micromanagement} (micro) and \emph{macromanagement}
(macro). Micro are the tactical actions and skills involving
individual battles between a game, while Macro are the strategic
decisions such as building management, resource usage, and unit
creation. A sucessful player (or bot) needs to eventually master
these two aspects of the game.

Development of AI agents for Starcraft is a vibrant research field
(See surveys by Onta\~non~\cite{OnSyUrRiChPr13} and
Certicky~\cite{CeCh17}). They highlight that, unlike Go and Chess,
Starcraft presents Game Playing AI the challenge of dealing with
real-time, partial information scenarios. Therefore, Starcraft is a
hard challenge for AI. An unofficial framework for the development of
AI agents exists in the form of the \emph{Brood Wars API}
(BWAPI)~\cite{BWAPI}. It allows the user to read and write to the
game's memory, allowing a wide range of agent types, from agents that
control limited parts of an AI opponent, to agents with the same range
of inputs and outputs as a human player.

% This requires good thinking, decision making and fast reactions.
% The game features three different races and all units are unique to
% their respective races, performs differently and requires different
% tactics.  The game is well known for being well balanced and as such
% was widely used in competitions.  Indeed even though each race is
% unique, has different strengths and abilities, their overall
% strength is the same and no race has an advantage over the
% other. This is because the game is now quite old and the balance
% have been polished over time via game updates provided by Blizzard
% Entertainment, developer and publisher of the game. It makes this
% game a great support to develop competitive, modular, adaptative
% agents.

% To create Starcraft: Broodwar agents, the \emph{BWAPI (Brood War
% Application Programming Interface) free and open source C++
% framework} is widely used by students, researchers and
% hobbyists. This framework isn't an official product from Blizzard
% and access and modify the game state by reading and writing directly
% the memory space of the game. As such is considered to be a "hack"
% that violates the End User License Agreement of the game. However,
% it is tolatered by company (and even encouraged given that they
% provided prizes to the tournament sponsorised by
% AIIDE\footnote{Artificial Intelligence and Interactive Digital
% Entertainment}). Reading / writing directly in the memory space of
% another program is considered to be an unsafe method (at least in
% term of stability) but the framework has been used and maintained
% for a long time and is considered stable.  It can be configured so
% as to only reveals the parts of the game state that are visibles by
% the agent, effectively providing only information that a real human
% player would have access to. That way, it is possible to write
% non-cheating AIs that operate under partial information conditions.

\subsection{Micromanagement}

In this study we focus on the \emph{Micromanagement} (Micro) aspect of
an Starcraft controller. Micro refers to the control of individual
units involved in a tactical combat. This involves sending commands to
each individual unit as a player would do, such as ``atack unit A'' or
``move to position X,Y''. 

{\bf TODO BENOIT: Add an interesting combat picture here}

There are many approaches to developing Starcraft micro
controllers. Wender and Watson investigate the viability of
Reinforcement Learning to this task~\cite{WeWa12}. They show that a
variety of RL techniques could achieve 100\% win ratios in small
combat situations, by controlling two actions: ``Attack'' and
``Retreat''.

Shantia et al. explore the use of neural networks to also order a unit
to ``attack'' or ``flee''~\cite{ShBeWi11}. The weights of the neural
network are adjusted by a RL algorithm. They show that you can train
the neural network on smaller combats (3x3) to perform well in larger
engagements (6x6).

More recently, Neuro Evolution has become a popular approach for
training Starcraft controllers. Zhen and Watson show a comparison of
NEAT and rtNEAT~\cite{ShWa13} for training a micro controller (also
using ``Attack'' and ``Retreat''). Their setup includes four classes
of engagement (ranged vs melee, melee vs melee, ranged vs ranged and
melee vs ranged), each with two teams of 12 units. They remark how the
unit composition of the match up has an effect on the variability of
the result.

One thing that all of these approaches (and a few others) have in
common is that the evaluation of the training is based on the win
ratio and the number of enemies killed. In this study, we want to
explore how to achieve a more ``humane'' behavior, i.e. achieving
victories while at the same time conserving the fighting force and
maximizing the number of surviving units.

\subsection{NEAT}\label{subsec:neat}

NEAT (\emph{Neuro Evolution of Augmented Topologies}) is a machine
learning technique which combines neural network and evolutionary
computation ideas to develop both the weights and the topology
(structure) of a neural network~\cite{StMi02}.

NEAT begins with a minimal topology, and grows it through training in
order to match the problem difficulty, adding and removing neurons and
connections and changing weights. Self connections and connections to
different layers are possible, which means that NEAT can evolve
recurrent neural networks. Additionally, NEAT includes a speciation
mechanism to promote genetic diversity in the population.

rtNEAT is a variant of NEAT for real time problems, which was
originally demonstrated on the combat game NERO. Since then, NEAT and
rtNEAT have been popular choices for the training of controllers in
tactical scenarios. The ability to match each unit to a separate
network in the population, and the ability to progressively learn more
complex behaviors are two compelling reasons for the use of
NEAT/rtNEAT in Starcraft.

Accordingly, recently variants of NEAT such as CascadeNEAT have also
been proposed to answer to specific problems.

\subsection{Cascade-NEAT}\label{subsec:cascade-neat}

Cascade-NEAT is a variant of NEAT which aims at restricting the neural
network search space to topologies that have a so-called cascade
architecture. A cascade architecture means that all hiden neurons are
connected to all other neurons, and only connections associated with
the most recently added hidden neuron are modified by the evolutionary
process.

Kohl~\cite{KoAS09,KoMi09} reports that this method is good for
fractured problems, such as the concentric spiral clustering problem
and the keepaway soccer decision problem. However, Cascade-NEAT
performs very poorly on some problems requiring recurrent connectivity
patterns, since it cannot produce such patterns. An example of such
failure is the non-Markovian double pole-balancing problem for which
cascade-NEAT doesn't find a single suitable solution whereas the
standard NEAT performs very well.

To our knowledge, Cascade-NEAT has not been applied to the Starcraft
problem. Because of the similarity of some strategic decisions
involved in Starcraft micro management and the keepaway soccer problem
we decided to include Cascade-NEAt as one of the alternatives for the
current investigation.

\subsection{Novelty search}\label{subsec:novelty-search}

Novelty Search is an approach that, instead of rewarding progressing toward a
fixed objective, \emph{rewards being different from others solutions and past ones}.
Usually, evolutionary algorithm measures how close the solution is to the goal
in order to perform the selection processus. Novelty Search, rather measure
distance (i.e.: behavioural, positional, …) between solutions to do that.
\emph{Sometimes not explicitly looking for the objective
leads to finding a solution faster}.
In particular in the case of \emph{deceptive problems}. Indeed, intermediate steps
may requires to evolve towards a totally different direction.
In that case, objective-oriented algorithm may stuck itself
in a local optima and cause the search to fail.
\citet{LeSt11} demonstrated this with an experiment where a robot controlled by a
neural network have to find a specific location in a maze.
A solution novelty is characterized by the end location of the robot
compared to all others. Only Novelty Search was able
to find the goal in the deceptive maze.

\subsection{Potential fields}\label{subsec:potential-fields}

Also called Artificial Potential Fields (APF), Potential Fields is a method
using \emph{attracting and repelling fields in the space} mainly used for maneuvering.
The sum of all the fields given the agent's position is used to determine the
direction for its next move. The more an agent is close to a field's center,
the more the said field's force is stronger and conversely.
It was originally used to make robots navigate between obstacles. Obstacles are
given a repelling field while the objective is given an attractive field. [source?]
However, using it in RTS games has been explored as well \cite{HaJo08}.
\citet{BoAu12} applied this method to StarCraft micromanagement along with
a multi-objective evolutionary algorithm called NSGA-II to optimize it automatically.



\section{Experimental Framework}\label{section:proposal}

In order to to execute the proposed study, the following framework
with three components was implemented. NEAT and the studied varients
were implemented as a C++ library. The bot that connects to BWAPI,
performs the learning and combat tasks, and logs the resulting data
was also implemented as a C++ project. The scripts for data analysis
were implemented as an R project. The source code and data files for
each of these components are available at the following repositories:
{\bf LIBRARYNAME\footnote{\url{https://github.com/uuuuu/nnnnn}}}, {\bf
  BOTNAME\footnote{\url{https://github.com/uuuuu/nnnnn}}}, and {\bf
  ANALYSISNAME\footnote{\url{https://github.com/uuuuu/nnnnn}}}

%NEToolKit\footnote{\url{https://github.com/CBenoit/NEToolKit}}
%Evolubot\footnote{\url{https://github.com/CBenoit/Evolubot}}.
%Evolubot_Analysis{\url{https://github.com/CBenoit/EvolubotAnalysis}}.

The four variants investigated in this study are as follows
\begin{itemize}
    \item \emph{Vanilla}. The default NEAT, as described in
      section~\ref{subsec:neat}. Each generation is evaluated using one
      game, where each unit uses the network of a different individual
      in the Population. The fitness of each individual is based on
      the performance of that particular unit.
    \item \emph{Cascade-NEAT}. The Cascade-NEAT variant, described in
      section~\ref{subsec:cascade-neat}. The evaluation procedure is
      the same as for Vanilla NEAT.
    \item \emph{Novelty Search with vanilla NEAT}. Here we use a
      different fitness function (described below) following the
      Novelty Search method described in
      section~\ref{subsec:novelty-search}. The evaluation procedure is
      the same as for Vanilla NEAT above.
    \item \emph{``Unified'' networks with vanilla NEAT}. In this
      variant, we modify the evaluation procedure. Only one individual
      is evaluated per game and, on each game, the entire squad has
      the same neural network. The fitness is calculated based on the
      performance of all individuals. For the sake of fairness, the
      number of generations is reduced proportionally in comparison
      with the above three variants.
\end{itemize}


\subsection{Neural Networks Inputs and Outputs}

The Neural Network used in our experimental framework has the
following inputs and outputs.

Inputs:
\begin{itemize}
    \item The cooldown of the ground weapon.
    \item The distance to the closest enemy;
    \item The number of enemies in sight;
    \item The distance to the closest allies;
    \item The numberof allies in sight;
    \item (Marines only) Whether the Unit has \emph{Stimpack} active or not.
\end{itemize}

Outputs:
\begin{itemize}
    \item {\bf Attack}: Attacks the weakest enemy in sight;
    \item {\bf Retreat}: Move away from enemies in sight;
    \item {\bf Spread out}: Move away from allies in sight;
    \item {\bf Gather}: Mote towards allies in sight;
    \item {\bf Use Stimpack}: For marines only.
\end{itemize}

The \emph{attack} action orders the unit to attack the weakest enemy
at fire range. If there are no enemies in fire range sight, the
unit move towards the closest visible enemy. Otherwise, it does nothing.

The \emph{retreat}, \emph{spread out} and \emph{gather} actions
calculate a weighted vector to decide the position that the unit is
ordered to move to. This vector is calculated based on the method used
by~\cite{Shantia11ConnectionistSC, Wender12ReinforcementMicroSC,
  Zhen13NeuroEvoSC}.

Our version differ slightly though. The weight of each unit to the
vector is proportional to their distance to the agent. In the case of
\emph{Retreat}, ally units get a significantly smaller weight than
enemy units, so that the agent avoid collision with allies while
moving away from enemies. For \emph{Spread Out}, only allies are
weighted. For \emph{Gather}, only allies are weighted and they have
negative weight.

\subsection{Fitness Specification}\label{subsec:fitness-specification}

In this study we consider that, during an actual game, the ability
to preserve's one's units is as important as the ability to defeat
enemy units. Therefore, we formulate a fitness function that
measures survival rates as well as victory rates.

For the Vanilla and Cascade variants, the fitness function is
given by:
\begin{equation}\label{eq:fitness_unit}
  (\text{damage\_dealt} \times A + \text{health} \times B +
  \text{survivors} \times C)^D,
\end{equation}
Where \emph{damage\_dealt} is the amount of damage caused to all
enemies, \emph{health} is the total remaining health of the unit, and
\emph{survivors} is the total remaining ally units, to promote
cooperation. $A, B, C,$ and $D$ are weight parameters (discussed in
the next section).

For the Unified variant, the fitness function is given by:
\begin{equation}\label{eq:fitness_group}
  (\text{damage\_dealt} \times A + \text{survivors} \times C)^D.
\end{equation}
Where the components are the same as in the previous equation.

Finally, for the Novelty variant, we use the difference in fitness as
the Novelty metric. Better alternative choices of Novelty metric are
discussed in section~\ref{section:conclusion}


\section{Matchups and experiments}

\begin{itemize}
    \item 22 marines vs 22 marines (with stimpacks)
    \item 22 marines vs 44 zergs (with stimpacks)
    \item 22 vultures vs 22 vultures
    \item 22 vultures vs 22 zealots
\end{itemize}

For each matchup and for each technique, we run three times an evolution from scratch up to the 100th generation.
Then, 50 games using the five best genomes produced during the evolution.

During the evolution the following data are logged:
\begin{itemize}
    \item number of survivors
    \item average fitness of each generation
    \item best fitness of each generation
    \item best current fitness ever for each generation
\end{itemize}

For the best units games, we only need to log the number of survivors.

We consider the number of survivors to be a relevent criteria. Indeed, a high number of survivors means the skirmish went well
and in a full game, these survivors can be reallocated to the next battle and can thus influence the long term outcome.
However, this information is not showed as much as we would like to. \citet{SiSuBa14} though also included some measurement of loss
and survivors in their evaluation.

Marines with stimpack may be a deceptive problem since it's basicaly losing health to gain performance.
We hope the agents learns to do so in order to win and increase survivors.

Especially, the marines vs zerg problem may be good to learn how to use stimpacks since it’s very difficult
for marines to win without using them. So, more stimpack usage is expected for this problem.

Novelty currently just use fitness as novelty metrics. Not best use case at all for this method.

Of course, as for all genetic algorithms, a lot of parameters has to be set up.

First, for the NEAT library in itself, it has being set up the following.

These are the overidden parameters especially for the experiments:

\begin{tabular}{rll}
    \toprule
    Parameter & Value & Tweaks \\
    \midrule
    Number of inputs & 5 & 6 for marines \\
    Number of outputs & 4 & 5 for marines \\
    Initial population size & 22  & — \\[1ex]

    Compatibility threshold & 2 & — \\
    Dynamic compatibility threshold & true & — \\
    Target number of species & 4 & — \\
    Keep same species' representant & true & — \\[1ex]

    Use best genomes library & true & false if novelty search \\
    Bad genome max fitness & 100 & 1000 if unified mode \\
    \bottomrule
\end{tabular}

The initial population size match the number of controlled units in the used scenarios.

The number of inputs and outputs is greater for marines because of the stimpack behaviour.
Marine neural networks can sense wether or not the stimpack is active on the unit in addition to all other inputs.
The new output had to be added to be used for stimppack activation.

The keep same species' representant parameter is used to avoid the update of the representant of species with a
random member at each generation as described in the original NEAT paper and instead keep the first representant of
the species.

The best genomes library is not used for novelty search because, the fitness is not used as a performance metrics in
that situation. Instead, a separate list of the best genomes is managed.

The bad genome max fitness determine which genomes are to be considered bad given the fitness. These genomes
may occasionally be replaced by one from the best genome library.

Other parameters are kept to defaults which, at least for the moment where the experiments where performed, were:

\begin{tabular}{rl}
    \toprule
    Parameter & Default value \\
    \midrule
    Mutation add link probability & 0.06 \\
    Mutation add neuron probabily & 0.03 \\
    Mutation remove neuron & 0.03 \\
    Mutation add cascade & 0.00 \\
    Mutation remove gene & 0.03 \\
    Mutation reenable gene & 0.03 \\
    Mutation toggle enable & 0.03 \\
    Mutation reset weights & 0.03 \\
    Mutation perturbate weights & 0.30 \\[1ex]

    Distance coefficient 1 & 1.0 \\
    Distance coefficient 2 & 1.5 \\
    Distance coefficient 3 & 0.4 \\
    Compatibility threshold change step & 0.05 \\[1ex]

    No reproduction threshold & 30 \\
    Mutation during crossover probability & 0.25 \\
    Initial weight perturbation & 10.0 \\
    Weight mutation power & 3.0 \\[1ex]

    Crossover probability & 0.20 \\
    Interspecies crossover probability & 0.05 \\
    Crossover ``multipoint random'''s weight & 2 \\
    Crossover ``multipoint best'''s weight & 0 \\
    Crossover ``multipoint average'''s weight & 0 \\[1ex]

    Replace bad genes using best genomes library prob & 0.05 \\
    Best genomes library max size & 5 \\
    \bottomrule
\end{tabular}

The distance coefficients are used in the compatibility distance formula which is
\[dist = c1 × \frac{E}{N} + c2 × \frac{D}{N} + c3 × W\]
with N the number of genes of the larger genome, E the number of excess genes, D the number of disjoint genes and
W the average weight differences of matching genes.
The is the original compatibility distance formula.~\cite{StMi02}

The ``no reproduction threshold'' parameter is used to prevent a species from reproduction if the
said species didn't improved for the given number of generations.

The only exception is for the cascade-NEAT variant where some mutation probabilities are changed:

\begin{tabular}{rl}
    \toprule
    Parameter & Default value \\
    \midrule
    Add link & 0.00 \\
    Add neuron & 0.00 \\
    Remove neuron & 0.00 \\
    Add cascade & 0.03 \\
    Remove gene & 0.00 \\
    \bottomrule
\end{tabular}

Indeed, the add cascade operation should be the only used in that case.

These parameters where mostly taken from the existing literature and were obtained through experimentation.

Next are the parameters of the Starcraft bot itself.

\begin{tabular}{rl}
    \toprule
    Parameter & Default value \\
    \midrule
    FRAMES PER UPDATE & 10 \\
    MAX DISTANCE NN & 1500 \\
    MAX ENTITY NN  & 20 \\[1ex]

    SURVIVAL PERF WEIGHT & 0.5 \\
    ATTACK PERF WEIGHT & 0.5 \\
    COOPERATIVE PERF WEIGHT & 10 \\
    UNIFIED ATTACK PERF WEIGHT & 0.33 \\
    UNIFIED COOPERATIVE PERF WEIGHT & 100 \\
    EXPONENT ON FITNESS & 1.3 \\
    \bottomrule
\end{tabular}

``MAX DISTANCE NN'' and ``MAX ENTITY NN'' are, respectively, the maximal distance and the maximal number of entities a neural network can perceive.
This is used for normalization purpose. All values greater than these will result in the value \(1\) being fed as the input so that
all inputs are in range \([0, 1]\).

The weights are used to model the fitness function. There is two fitness function according to whether unified mode is active or not.
If not, the function is given by:
\[(damages\_dealt × A + health × B + survivors × C)^D\]
where A is the attack performance weight, B the survival performance weight, C the
cooperative performance weight and D the exponent on fitness.
If, however, it is in unified mode, the following function is instead used:
\[(damages\_dealt × A + survivors × C)^D\]
where A is the unified attack performance weight, C the unified cooperative performance weight
and D the same as previously exponent on fitness.

The number of survivors is integrated in the fitness function so as to encourage cooperation and survival rate.



\section{Experimental Results}\label{section:experiments-results}

\subsection{Quantitative Results}\label{subsec:quantitative}

As discussed in the previous session, the four NEAT variants were
compared in two ways: First we perform three full evolution runs for
each variant/matchup pairing, and then the five best genomes obtained
from this training are run 50 times to evaluate their victory and
survival ratios. A summary of these results is presented in
Table~\ref{table:quantitative}.

\begin{table}
    \caption{Victory Ratio and Median Survival for the best genomes.
            Survival values only consider victory matches.}
    \label{table:quantitative}
    \begin{tabular}{llll}
        \toprule
        Variant & Matchup & Win Ratio & Survival \\
        \midrule
        Vanilla & marine/marine & 0.77 & 8 \\
        Unified & marine/marine & 0.77 & 8 \\
        Cascade & marine/marine & 0.85 & 8 \\
        Novelty & marine/marine & 0.55 & 8 \\[1ex]

        Vanilla & marine/zergling & 0.03 & 4 \\
        Unified & marine/zergling & 0.01 & 10 \\
        Cascade & marine/zergling & 0.05 & 5 \\
        Novelty & marine/zergling & 0.09 & 10 \\[1ex]

        Vanilla & vulture/vulture & 0.92 & 8 \\
        Unified & vulture/vulture & 0.90 & 10 \\
        Cascade & vulture/vulture & 0.93 & 8 \\
        Novelty & vulture/vulture & 0.93 & 9 \\[1ex]

        Vanilla & vulture/zealot & 0.87 & 8 \\
        Unified & vulture/zealot & 0.93 & 9 \\
        Cascade & vulture/zealot & 0.92 & 9 \\
        Novelty & vulture/zealot & 0.99 & 9 \\
        \bottomrule
    \end{tabular}
\end{table}

% \subsubsection{Victory Rate Analysis}

From this table, we can observe that in general the methods managed to
learn a good tactic to defeat their enemies, with the exception of the
marine/zergling matchup.
%\footnote{This could be considered scientific evidence that Blizzard needs to nerf zerglings}
Small variations are observed on the methods within each matchup, but
an ANOVA test confirms that these variations are not significative (F
= 0.85).

% \subsubsection{Survival Rate Analysis}

In the same fashion, it was not possible to observe a clear difference
among the variants in terms of the median number of survivors in
winning matches. Figure~\ref{fig:survivors} shows the survivor numbers
for each variation/matching pairing.

\begin{figure}
  \includegraphics[width=.5\textwidth]{figures/survivors}
  \caption{Box plot for the survival ratios}\label{fig:survivors}
\end{figure}

% \subsubsection{Evolution Rate Analysis}


\begin{figure*}
  \includegraphics[width=.4\textwidth]{figures/evolution_zealot}
  \includegraphics[width=.4\textwidth]{figures/evolution_unified}
  \caption{Training progress for two vulture vs zealot matchups. Left:
    Cascade NEAT vs zealots. Right: Unified NEAT. Note that the
    fitness function are different so absolute values should not be
    compared directly.}\label{fig:evolution}
\end{figure*}

Finally, we observed the training progress of the
experiment. Figure~\ref{fig:evolution} show two examples of training
graphs that are representative of the experiment as a
whole\footnote{the figures for all matchings can be found at the
online repository}. In the Cascade NEAT training (left side) we see
that while there is a small increase in best and mean fitness, there
are large variations every generation. In the Unified NEAT training
(right side) these variations are not present. In unified neat, all
networks in the same combat experiment share the same fitness result,
while in the other variants (such as the Cascade Neat) each network in
the same combat is evaluated separately, even though they influence
each other (a weak network will make the overall combat harder for a
stronger network).

However, while this difference is observable in the training curve, as
we see in Figure~\ref{fig:evolution}, it did not generate an
observable difference in the final result
(Table~\ref{table:quantitative}), which is interesting.

\subsection{Qualitative analysis}\label{subsec:qualitative}

From the quantitative analysis we learned that a winning strategy was
generally found (high winning ratio), while the differences among
variants was not strong. In particular, the survivor rate was the same
across methods with the exception of the particularly hard
marine/zergling matchup.

To better understand this lack of difference, and what could be done
to increase the survival ratio, we performed a qualitative analysis of the
results of the experiment by studying the different behaviors that emerged.

\subsubsection{Developping stimpack behaviour}

\begin{figure}
    \includegraphics[width=.5\textwidth]{figures/marines_vs_zerglings_combat}
    \caption{Marines fighting against zerglings {\bf FIXME: really interesting to show?}}\label{fig:marines_vs_zerglings}
\end{figure}

Standard NEAT doesn’t seems to be good a developing the stimpack behaviour since organisms
that survives without stimpack get better reward,
even though they may have survived because others used stimpacks to get in the frontline
and got themselves killed first.
So, organisms tends to lose this behaviour over generations.
Cascade-NEAT did similar to standard NEAT.

In comparison, the unified version, which is good a developing uniform behaviours
was somewhat better at developing this behaviour.
Indeed, since every agents use the same genotype, all units in the round
has the behaviour encoded and didn't win because agents based on a different genome
went on the frontline first and got a bad fitness doing so.

Finally, with novelty search, the behaviour is lost less easily and some of the best genomes still use it.
But it's not efficient enough to win easily especially during evolution where some other units
simply flee. It might however explains the sligthly better results saw with the statistical approach.

For each technique, the major reason why the stimpack behaviour is lost is because the fighting behaviour along
the stimpack behaviour is sometimes poor. They try to flee without reason or just don't attack even when the
stimpacks was activated at an appropriate moment.
So it seems possible to get a good stimpack behaviour especially with novelty and a better novelty measurement
that doesn't use the fitness which doesn't reflect the quality of the stimpack behaviour directly.

The marines vs marines problem is solved efficiently just by waiting for the
enemy and attacking all out mostly without stimpacks and it's precisely the convergent behaviour.
Depending on the enemy’s formation at it’s arrival it can be even more effective.
They can indeed shoot first with more firepower.

\subsubsection{Kiting behaviour}

\begin{figure}
    \includegraphics[width=.5\textwidth]{figures/vultures_kiting_screenshot}
    \caption{Vultures using kiting technique to beat zealots}\label{fig:vultures_kiting}
\end{figure}

All the presented method performs exceedingly well on the kiting problem (that is vultures vs zealots).
This is confirmed by the statistical results shown in subsection~\ref{subsec:quantitative}. After a short amount of generations, adequate
individuals start to emerge (even with the first species, meaning the topology required isn’t complex at all)
and after a few more generations, networks weights are tweaked to get a quite decent kiting behaviour with any
technique, the vulture is able to maintain a good distance between itself and the enemies as shown on
Figure~\ref{fig:vultures_kiting}. In some cases, vultures are even able to shoot without
losing speed and thus withdraw really quickly from risks. This ressemble
a classical vulture micro technique called “patrol micro” involving the use of the so-called patrol command's
lack of cooldown to have the vulture move, fire, and retreat without slowing down. We don’t really know how it
is achieved here without the patrol command, but it is effectively the same result.

It is interesting to note that for the matchup vultures vs vultures, our bot was even more effective by avoiding opponent's
projectiles using, again, effective kiting. Even against a ranged opponent it can shines (though here the projectives have
some physical properties unlike marines's bullets).

To improve: friendly collisions blocking effective retreat.

\subsubsection{Spreading out behaviour}\label{subsec:spreading}

\begin{figure}
    \includegraphics[width=.15\textwidth]{figures/spreading_behaviour_standard_neat}
    \includegraphics[width=.15\textwidth]{figures/spreading_behaviour_unified}
    \caption{Spreading out behaviour with standard NEAT (left) and unified version (right)}\label{fig:spreading-behaviour}
\end{figure}

With the vultures, another interesting behaviour was the vultures spreading out to cover large space
(as showed on Figure~\ref{fig:spreading-behaviour}
and then converge back to the enemy once it's spotted. However, this behaviour tends to disappear as
well probably because oftenly the fitness is not as good for such individuals for various reasons.
It may be because spreading out and converging back takes time and lead to less damages inflicted
or because some individuals had the spreading behaviour without attacking afterward, especially
for the unified version where the behaviour is often lost very quickly because in such case not
even a single unit try to attack.
Moreover, we didn't saw the behaviour at all with Cascade-NEAT, but it may be due to randomness.

\begin{figure}
    \includegraphics[width=.15\textwidth]{figures/spreading_behaviour_novelty}
    \caption{Spreading out behaviour with Novelty}\label{fig:spreading-behaviour-novelty}
\end{figure}

That being said, novelty search found a very efficient spreading behaviour leading to a complete
encirclement of the oponent as show by Figure~\ref{fig:spreading-behaviour-novelty}.



\section{Discussion}\label{section:discussion}

TODO



\section{Conclusion}\label{section:conclusion}

This paper presented a comparison between several methods related to NEAT:
\emph{standard NEAT, cascade-NEAT and NEAT with novelty search} where the whole
population is evaluated at once in one game as well as \emph{``unified'' NEAT} where
in one game only one individual whose neural network is used for every unit is evaluated
instead of evaluating all the population.
To do so, an open source C++ library for NEAT and a Starcraft Bot using it was developed.
The default AI was used to measure performance of the Bot by playing against it.
Four matchups were considered: marines vs marines, marines vs zerglings, vultures vs vultures
and vultures vs zealots.

While no statistical difference has been found, behaviours can differs a little bit as seen
on Subsection~\ref{subsec:qualitative}. Novelty search was able to keep longer the
stimpack behaviour and did great at encircling the oponent with the spread out behaviour.
Cascade-NEAT didn't developed the spread out behaviour at all in our observations.
Unified version developped a spread out behaviour but didn't used it to perform any efficient attack.
Other than that, behaviours were also quite similar, especially the kiting behaviour is
successfully developed by every methods in a very satisfactory way. The similarity in behaviours may
be due to the low granularity of the inputs and outputs of the neural networks.

As explained in subsection~\ref{subsec:fitness-specification} the novelty metric is given by
the fitness of the individual which is not ideal. A better novelty measurement would directly reflect
the behaviour of the units such the way the agent move or the way it performs kiting. Whereas the fitness
function can only reflect the behaviour indirectly. Finding a good novelty measurement can be a
very difficult problem depending on the application, and Starcraft is one of them.



\section*{Acknowledgements}

AUTHOR 1 was partially funded by ACKNOWLEDGEMENT 1.

\bibliographystyle{ACM-Reference-Format}
\bibliography{bibliography}{}

\end{document}

