\section{Literature review}\label{section:literature-review}

\subsection{NEAT}

NEAT: NeuroEvolution method. Starts from a minimal topology and grows it to match the problem difficulty.

Article: neuroevolution through augmenting topologies, 2002.

\subsection{Cascade-NEAT}

Restrict the search process to topologies that have a cascade architecture, good for fractured problems.

Article: learning in fractured problems with constructive neural networks algorithms.

Article: evolving neural networks for strategic decision-making problems.

\subsection{Novelty search}

\emph{explanation todo} \citet{LeSt11}

\subsection{Others}

Applying reinforcement learning to small scale combat in the strategy game starcraft broodwar. Uses Q-learning and Sarsa.

Connectionist reinforcement learning for intelligent unit micro management in starcraft. Uses neural networks to evaluate Q-values for Sarsa reinforcement learning method. Results on incremental learning. Interesting part for reward units: dead units don’t get a negative reward for dying. Instead the reward of the dead unit is the average reward of all living units in the next state. It introduce a sort of cooperative behaviour. See neural-fitted Q-iteration.

Micromanagement potential fields multi-objectives genetic algorithm. Micromanagement using Potential Fields tuned with a Multi-Objective Optimized Evolutionary Algorithm (namely NSGA-II). Present a model to handle priority targets, focus firing, retreat spots, … Talk about how to get a behaviour optimizing multiple objectives. Use of the killscore.

NeuroEvolution for micromanagement in rts game starcraft bw. About battle micromanagement and NEAT vs rtNEAT.

