\section{Background}\label{section:literature-review}

\subsection{Starcraft}

{\bf about Starcraft:} Talk about starcraft the game, add a recent
survey on starcraft research

{\bf about BWAPI:} Talk about the technical side, API, bot code, etc.

\subsection{Micromanagement}

Talk about the micromanagement problem, give some examples, show a
match picture

{\bf recent works about micromanagement in starcraft:} Add some recent
works about micro in starcraft, and comment how they all measure
success strictly as the winning rate of encounters, and how in this
work we want to extend this area by taking the number of unit lost
into account.

\subsection{NEAT}

NEAT is a \emph{NeuroEvolution} method. The principle is to
\emph{start with a minimal topology and grow it} to match the problem
difficulty in order to find an appropriate network.  It does that by
adding and removing neurons, changing weights, adding connections, ...
Any neuron can be connected to any neuron including itself, meaning
even \emph{recurrent neural networks} can be generated.  At each
generation high fitness networks get better chance of
reproduction. But even so, \emph{NEAT maintains genetic diversity}
through a process called \emph{speciation}. That is, similar networks
are considered to be in the same species. Then, to encourage
innovation, \emph{explicit fitness sharing} is performed on members of
the same species. Each species is assigned a number of offsprings it
can produce. Then species' members compete against each
others. \cite{StMi02}

Various variant of the method has been created to answer specific
problems such as \emph{Cascade-NEAT}.

\subsection{Cascade-NEAT}

Restrict the search process to topologies that have a cascade
architecture, good for fractured problems.

Article: learning in fractured problems with constructive neural networks algorithms.

Article: evolving neural networks for strategic decision-making problems.

\subsection{Novelty search}

\emph{explanation todo} \cite{LeSt11}

\subsection{Potential fields}

Micromanagement potential fields multi-objectives genetic
algorithm. Micromanagement using Potential Fields tuned with a
Multi-Objective Optimized Evolutionary Algorithm (namely
NSGA-II). Present a model to handle priority targets, focus firing,
retreat spots, … Talk about how to get a behaviour optimizing multiple
objectives. Use of the killscore.

\citet{SiSuBa14}

\subsection{Others}

Applying reinforcement learning to small scale combat in the strategy
game starcraft broodwar. Uses Q-learning and Sarsa.

Connectionist reinforcement learning for intelligent unit micro
management in starcraft.  Uses neural networks to evaluate Q-values
for Sarsa reinforcement learning method. Results on incremental
learning.  Interesting part for reward units: dead units don’t get a
negative reward for dying.  Instead the reward of the dead unit is the
average reward of all living units in the next state.  It introduce a
sort of cooperative behaviour. See neural-fitted Q-iteration.

NeuroEvolution for micromanagement in rts game starcraft bw. About
battle micromanagement and NEAT vs rtNEAT.

