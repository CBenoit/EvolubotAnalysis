\section{Proposal}\label{section:proposal}

A fresh C++ implementation of these methods was developed and is hosted on Github under the name of NEToolKit\footnote{\url{https://github.com/CBenoit/NEToolKit}}.

The Starcraft bot can be found on Github as well under the name of Evolubot\footnote{\url{https://github.com/CBenoit/Evolubot}}.

The analysis of the data was performed with the R language. Source can, again, be found on Github\footnote{\url{https://github.com/CBenoit/EvolubotAnalysis}}.

We want to evualate several techniques that are the following:
\begin{itemize}
    \item \emph{Vanilla NEAT}. The default NEAT, as described in Sub-Section~\ref{subsec:neat}. One generation is evaluated with one game
        where every unit is an individual with its own neural network. Therefore, the fitness is individual on a per unit basis.
    \item \emph{Cascade-NEAT}. The cascade-NEAT variant, as described in Sub-Section~\ref{subsec:cascade-neat}.
        The evaluation mothed is the same as for vanilla NEAT above.
    \item \emph{Novelty Search with vanilla NEAT}. The novelty search method as described in Sub-Section~\ref{subsec:novelty-search} applied to the NEAT algorithm.
        The evaluation methed is the same as for vanilla NEAT above.
    \item \emph{``Unified'' networks with vanilla NEAT}. For this one, as opposed to the evaluation method explained above, only one indivual is
        evaluated per game and thus to evaluate a whole generation, a number of games equal to the population size must be processed.
        The fitness of an indivual is computed by how well all the controlled units performs. Each unit still has its own neural network
        but, they're all obtained by using the genome of the individual being evualated.
\end{itemize}

