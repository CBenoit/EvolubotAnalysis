\section{Proposal}\label{section:proposal}

A fresh C++ implementation of these methods was developed and is hosted on Github under the name of LIBRARY
NAME\footnote{\url{https://github.com/uuuuu/nnnnn}}.%NEToolKit\footnote{\url{https://github.com/CBenoit/NEToolKit}}.

The Starcraft bot can be found on Github as well under the name of
BOT NAME\footnote{\url{https://github.com/uuuuu/nnnnn}}.%Evolubot\footnote{\url{https://github.com/CBenoit/Evolubot}}.

The analysis of the data was performed with the R language. Sources can, again,
be found on Github\footnote{\url{https://github.com/uuuuu/nnnnn}}.%\footnote{\url{https://github.com/CBenoit/EvolubotAnalysis}}.

We want to evaluate several techniques that are the following:
\begin{itemize}
    \item \emph{Vanilla NEAT}. The default NEAT, as described in Sub-Section~\ref{subsec:neat}. One generation is evaluated with one game
        where every unit is an individual with its own neural network. Therefore, the fitness is individual on a per unit basis.
    \item \emph{Cascade-NEAT}. The cascade-NEAT variant, as described in Sub-Section~\ref{subsec:cascade-neat}.
        The evaluation mothed is the same as for vanilla NEAT above.
    \item \emph{Novelty Search with vanilla NEAT}. The novelty search method as described in Sub-Section~\ref{subsec:novelty-search} applied to the NEAT algorithm.
        The evaluation methed is the same as for vanilla NEAT above.
    \item \emph{``Unified'' networks with vanilla NEAT}. For this one, as opposed to the evaluation method explained above, only one indivual is
        evaluated per game and thus to evaluate a whole generation, a number of games equal to the population size must be processed.
        The fitness of an indivual is computed by how well all the controlled units performs. Each unit still has its own neural network
        but, they're all obtained by using the genome of the individual being evualated.
\end{itemize}

We consider that in an actual game, being able to preserve one's
units can be as important as being able to defeat the enemy units.
Therefore, in this study we will measure survival rates as well.

\subsection{Neural Networks Inputs and Outputs}

For all the matchups, we will generate networks with at least the following inputs:
\begin{itemize}
    \item The \emph{cooldown} of the ground weapon.
    \item The \emph{distance} to the \emph{closest enemy}.
    \item The \emph{number} of enemies \emph{in sight}.
    \item The \emph{distance} to the \emph{closest allies}.
    \item The \emph{number} of allies \emph{in sight}.
\end{itemize}

And the following outputs:
\begin{itemize}
    \item \emph{Attack} the weakest enemy in sight.
    \item \emph{Retreat} from the battlefield: \emph{draw away from enemies in sight}.
    \item \emph{Spread out}: \emph{draw away from allies in sight}.
    \item \emph{Gathering}: get close to \emph{allies in sight}.
\end{itemize}

In addition, marines get one additional input and output:
\emph{whether it is stimmed or not} as the input and \emph{use a stimpack} as the output.

\subsection{Fitness Specification}

The weights are used to model the fitness function. There are two
fitness functions, according to whether the Unified NEAT variant is
being used or not.  The regular fitness function is given by
\begin{equation*}
  (\text{damages\_dealt} \times A + \text{health} \times B + \text{survivors} \times C)^D,
\end{equation*}
while the fitness for the Unified NEAT variant is
\begin{equation*}
  (\text{damages\_dealt} \times A + \text{survivors} \times C)^D.
\end{equation*}
Here \emph{A} is the attack performance weight, \emph{B} the survival
performance weight, C is the cooperative performance weight and D is
the exponent on the fitness.

To encourage the variants to find strategies that promote cooperation
and survival, we have included the number of survivors as a component
of the fitness function.

Novelty currently just use fitness as novelty metrics. Not best use case at
all for this method.

