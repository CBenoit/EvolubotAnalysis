\section{Introduction}\label{section:introduction}

\emph{Starcraft: Brood Wars} is a Real Time Strategy (RTS) computer
game that has in recent years captured the attention of researchers on
game-playing artificial intelligence. StarCraft is a complex game that
can be studied from many different points of view: Strategic Planning,
Execution of tactical maneuvers, Estimation of hidden information
about the opponent, etc.

In this work, we focus on the problem of ``micro'' in
Starcraft. Micro, short for micromanagement, is the problem of
directly controlling a small number of units directly in an engagement
with units from an opponent player. The tasks required in micro
include moving damaged units out of the fire range from the enemy,
spreading the units to maximize the fire arc, and using an unit's
special abilities at the right time. Also, the decisions for each of
these tasks will differ according to the units controlled by the
player and the opponent.

Learning approaches for micro in Starcraft and other similar games has
been studied by many groups recently. For example, Shantia et
al.~\cite{ShBeWi11} created a micro controller focused around a neural
network with weights adjusted by Sarsa; Wender and
Watson~\cite{WeWa12} explored Q-learning and Sarsa; Zhen and
Watson~\cite{ShWa13} compare NEAT and rtNEAT; and Liu et
al~\cite{SiSuBa14} influence maps and potential fields.  Although each
of these works uses a different algorithm to generate the micro
controller, one thing that they all have in common is that they mostly
measure success as the win rate, i.e. the total number of combats won
(although Liu et al. does include some measure of unit loss and
survivors in their evaluation).

In this work we want to explore how to evolve a micro controller that
is able to not only achieve a high win rate, but also do so with as
few losses as possible. In an actual game, it is important for a player
to preserve their own forces, so that they can reallocate the remaining
forces to future battle, and compound their advantage through the match.

To achieve this, we propose an evaluation function that takes into
account the damage taken and the number of surviving units at the end
of a match, and compare four different NEAT implementations using this
utility function, as well as Novelty Search. We analyze the results of
four Match ups (Marines vs Marines, Marines vs Zerg, Vultures vs
Vultures and Vultures vs Zealots). Quantitatively, none of the
implemented variants showed a strong difference from current
results. But a qualitative analysis indicate directions for a more
humane micromanagement controller.
