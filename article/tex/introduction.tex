\section{Introduction}\label{section:introduction}

\emph{Starcraft: Brood Wars} is a Real Time Strategy (RTS) computer
game that has in recent years captured the attention of researchers on
game-playing artificial intelligence. StarCraft is a complex game that
can be studied from many different points of view: Strategic Planning,
Execution of tactical manuevers, Estimation of hidden information
about the opponent, etc.

In this work, we focus on the problem of ``micro'' in
Starcraft. Micro, short for micromanagement, is the problem of
directly controlling a small number of units directly in an engagement
with units from an opponent player. The tasks required in micro
include moving damaged units out of the fire range from the enemy,
spreading the units to maximize the fire arc, and using an unit's
special abilities at the right time. Also, the decisions for each of
these tasks will differ according to the units controlled by the
player and the opponent.

Micro has been studied by many groups recently, such as {\bf TODO,
  include some papers and their tools here}.

In these studies, a successful micro technique is usually measured by
the ``win rate'', defined as the percentage of pre-defined encounters
where all the enemy units have been destroyed by the player
units. However, in an actual game, being able to preserve your own
units can be as important as being able to defeat the enemy
units. Therefore, in this study we {\bf TODO...}



%%%%%%%%%%%%%%%%%%%%%%%%%%%%%%%%%%%%%%%%%%%%%%%%%%%%%%%%%%%%%%%%%%%%%%%%%%%%%
% Moved this from the ``Experiment Design'' section, remember to rewrite later
We consider the number of survivors to be a relevent criteria.
Indeed, a high number of survivors means the skirmish went well
and in a full game, these survivors can be reallocated to the
next battle and can thus influence the long term outcome.
However, this information is not showed as much as we would like to. \citet{SiSuBa14}
though also included some measurement of loss
and survivors in their evaluation.
