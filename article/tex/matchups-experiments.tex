\section{Matchups and experiments}

\begin{itemize}
    \item 22 marines vs 22 marines (with stimpacks)
    \item 22 marines vs 44 zergs (with stimpacks)
    \item 22 vultures vs 22 vultures
    \item 22 vultures vs 22 zealots
\end{itemize}

For each matchup and for each technique, we run three times an evolution from scratch up to the 100th generation.
Then, 50 games using the five best genomes produced during the evolution.

During the evolution the following data are logged:
\begin{itemize}
    \item number of survivors
    \item average fitness of each generation
    \item best fitness of each generation
    \item best current fitness ever for each generation
\end{itemize}

For the best units games, we only need to log the number of survivors.

We consider the number of survivors to be a relevent criteria. Indeed, a high number of survivors means the skirmish went well
and in a full game, these survivors can be reallocated to the next battle and can thus influence the long term outcome.
However, this information is not showed as much as we would like to. \citet{SiSuBa14} though also included some measurement of loss
and survivors in their evaluation.

To encourage cooperation and survival rate, we integrate the number of survivors in the fitness function.

Marines with stimpack may be a deceptive problem since it's basicaly losing health to gain performance.
We hope the agents learns to do so in order to win and increase survivors.

Especially, the marines vs zerg problem may be good to learn how to use stimpacks since it’s very difficult
for marines to win without using them. So, more stimpack usage is expected for this problem.

Novelty currently just use fitness as novelty metrics. Not best use case at all for this method.

TODO: add all parameters for experiments.

