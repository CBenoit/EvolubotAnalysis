\section{Matchups and experiments}

\begin{itemize}
    \item 22 marines vs 22 marines (with stimpacks)
    \item 22 marines vs 44 zergs (with stimpacks)
    \item 22 vultures vs 22 vultures
    \item 22 vultures vs 22 zealots
\end{itemize}

For each matchup and for each technique, we run three times an evolution from scratch up to the 100th generation.
Then, 50 games using the five best genomes produced during the evolution.

During the evolution the following data are logged:
\begin{itemize}
    \item number of survivors
    \item average fitness of each generation
    \item best fitness of each generation
    \item best current fitness ever for each generation
\end{itemize}

For the best units games, we only need to log the number of survivors.

We consider the number of survivors to be a relevent criteria. Indeed, a high number of survivors means the skirmish went well
and in a full game, these survivors can be reallocated to the next battle and can thus influence the long term outcome.
However, this information is not showed as much as we would like to. \citet{SiSuBa14} though also included some measurement of loss
and survivors in their evaluation.

Marines with stimpack may be a deceptive problem since it's basicaly losing health to gain performance.
We hope the agents learns to do so in order to win and increase survivors.

Especially, the marines vs zerg problem may be good to learn how to use stimpacks since it’s very difficult
for marines to win without using them. So, more stimpack usage is expected for this problem.

Novelty currently just use fitness as novelty metrics. Not best use case at all for this method.

Of course, as for all genetic algorithms, a lot of parameters has to be set up.

First, for the NEAT library in itself, it has being set up the following.

These are the overidden parameters especially for the experiments:

\begin{tabular}{rll}
    \toprule
    Parameter & Value & Tweaks \\
    \midrule
    Number of inputs & 5 & 6 for marines \\
    Number of outputs & 4 & 5 for marines \\
    Initial population size & 22  & — \\[1ex]

    Compatibility threshold & 2 & — \\
    Dynamic compatibility threshold & true & — \\
    Target number of species & 4 & — \\
    Keep same species' representant & true & — \\[1ex]

    Use best genomes library & true & false if novelty search \\
    Bad genome max fitness & 100 & 1000 if unified mode \\
    \bottomrule
\end{tabular}

The initial population size match the number of controlled units in the used scenarios.

The number of inputs and outputs is greater for marines because of the stimpack behaviour.
Marine neural networks can sense wether or not the stimpack is active on the unit in addition to all other inputs.
The new output had to be added to be used for stimppack activation.

The keep same species' representant parameter is used to avoid the update of the representant of species with a
random member at each generation as described in the original NEAT paper and instead keep the first representant of
the species.

The best genomes library is not used for novelty search because, the fitness is not used as a performance metrics in
that situation. Instead, a separate list of the best genomes is managed.

The bad genome max fitness determine which genomes are to be considered bad given the fitness. These genomes
may occasionally be replaced by one from the best genome library.

Other parameters are kept to defaults which, at least for the moment where the experiments where performed, were:

\begin{tabular}{rl}
    \toprule
    Parameter & Default value \\
    \midrule
    Mutation add link probability & 0.06 \\
    Mutation add neuron probabily & 0.03 \\
    Mutation remove neuron & 0.03 \\
    Mutation add cascade & 0.00 \\
    Mutation remove gene & 0.03 \\
    Mutation reenable gene & 0.03 \\
    Mutation toggle enable & 0.03 \\
    Mutation reset weights & 0.03 \\
    Mutation perturbate weights & 0.30 \\[1ex]

    Distance coefficient 1 & 1.0 \\
    Distance coefficient 2 & 1.5 \\
    Distance coefficient 3 & 0.4 \\
    Compatibility threshold change step & 0.05 \\[1ex]

    No reproduction threshold & 30 \\
    Mutation during crossover probability & 0.25 \\
    Initial weight perturbation & 10.0 \\
    Weight mutation power & 3.0 \\[1ex]

    Crossover probability & 0.20 \\
    Interspecies crossover probability & 0.05 \\
    Crossover ``multipoint random'''s weight & 2 \\
    Crossover ``multipoint best'''s weight & 0 \\
    Crossover ``multipoint average'''s weight & 0 \\[1ex]

    Replace bad genes using best genomes library prob & 0.05 \\
    Best genomes library max size & 5 \\
    \bottomrule
\end{tabular}

The distance coefficients are used in the compatibility distance formula which is
\[dist = c1 × \frac{E}{N} + c2 × \frac{D}{N} + c3 × W\]
with N the number of genes of the larger genome, E the number of excess genes, D the number of disjoint genes and
W the average weight differences of matching genes.
The is the original compatibility distance formula.~\cite{StMi02}

The ``no reproduction threshold'' parameter is used to prevent a species from reproduction if the
said species didn't improved for the given number of generations.

The only exception is for the cascade-NEAT variant where some mutation probabilities are changed:

\begin{tabular}{rl}
    \toprule
    Parameter & Default value \\
    \midrule
    Add link & 0.00 \\
    Add neuron & 0.00 \\
    Remove neuron & 0.00 \\
    Add cascade & 0.03 \\
    Remove gene & 0.00 \\
    \bottomrule
\end{tabular}

Indeed, the add cascade operation should be the only used in that case.

These parameters where mostly taken from the existing literature and were obtained through experimentation.

Next are the parameters of the Starcraft bot itself.

\begin{tabular}{rl}
    \toprule
    Parameter & Default value \\
    \midrule
    FRAMES PER UPDATE & 10 \\
    MAX DISTANCE NN & 1500 \\
    MAX ENTITY NN  & 20 \\[1ex]

    SURVIVAL PERF WEIGHT & 0.5 \\
    ATTACK PERF WEIGHT & 0.5 \\
    COOPERATIVE PERF WEIGHT & 10 \\
    UNIFIED ATTACK PERF WEIGHT & 0.33 \\
    UNIFIED COOPERATIVE PERF WEIGHT & 100 \\
    EXPONENT ON FITNESS & 1.3 \\
    \bottomrule
\end{tabular}

``MAX DISTANCE NN'' and ``MAX ENTITY NN'' are, respectively, the maximal distance and the maximal number of entities a neural network can perceive.
This is used for normalization purpose. All values greater than these will result in the value \(1\) being fed as the input so that
all inputs are in range \([0, 1]\).

The weights are used to model the fitness function. There is two fitness function according to whether unified mode is active or not.
If not, the function is given by:
\[(damages\_dealt × A + health × B + survivors × C)^D\]
where A is the attack performance weight, B the survival performance weight, C the
cooperative performance weight and D the exponent on fitness.
If, however, it is in unified mode, the following function is instead used:
\[(damages\_dealt × A + survivors × C)^D\]
where A is the unified attack performance weight, C the unified cooperative performance weight
and D the same as previously exponent on fitness.

The number of survivors is integrated in the fitness function so as to encourage cooperation and survival rate.

