\section{Conclusion}\label{section:conclusion}

This paper presented a comparison between several methods related to NEAT:
\emph{standard NEAT, cascade-NEAT and NEAT with novelty search} where the whole
population is evaluated at once in one game as well as \emph{``unified'' NEAT} where
in one game only one individual whose neural network is used for every unit is evaluated
instead of evaluating all the population.
To do so, an open source C++ library for NEAT and a Starcraft Bot using it was developed.
The default AI was used to measure performance of the Bot by playing against it.
Four matchups were considered: marines vs marines, marines vs zerglings, vultures vs vultures
and vultures vs zealots.

While no statistical difference has been found, behaviours can differs a little bit as seen
on Subsection~\ref{subsec:qualitative}. Novelty search was able to keep longer the
stimpack behaviour and did great at encircling the oponent with the spread out behaviour.
Cascade-NEAT didn't developed the spread out behaviour at all in our observations.
Unified version developped a spread out behaviour but didn't used it to perform any efficient attack.
Other than that, behaviours were also quite similar, especially the kiting behaviour is
successfully developed by every methods in a very satisfactory way. The similarity in behaviours may
be due to the low granularity of the inputs and outputs of the neural networks.

A explained in subsection~\ref{subsec:fitness-specification} the novelty metric is given by
the fitness of the individual which is not ideal. A better novelty measurement would directly reflect
the behaviour of the units such the way the agent move or the way it performs kiting. Whereas the fitness
function can only reflect the behaviour indirectly. Finding a good novelty measurement can be a
very difficult problem depending on the application, and Starcraft is one of them.

